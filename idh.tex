\RequirePackage[l2tabu, orthodox]{nag}
\documentclass[french]{beamer}
\input{preamble/packages}
\input{preamble/redac}
\input{preamble/math_basics}
\input{preamble/math_mine}
\input{preamble/draw}

%\setbeamertemplate{headline}[singleline]
%\setbeamertemplate{footline}[authortitle]

\title[IDH]{L’indice de développement humain}
\subtitle{Histoire, définitions, critiques et extensions}
\subject{HDI}
\keywords{IHDI, PHDI}
\author[Ouissem, Morgane, Olivier]{Ouissem Chikh \and Morgane Le Plomb \and Olivier Cailloux}
\institute[Éco du dev, L3]{Économie du développement, L3}
\date{\formatdate{30}{9}{2024}}

\begin{document}
\begin{frame}[plain]
	\tikz[remember picture,overlay]{
		\path (current page.south west) node[anchor=south west, inner sep=0] {
			\includegraphics[height=8mm]{Dauphine-Noir.png}
		};
		\path (current page.south) ++ (0, 4em) node[anchor=south, inner sep=0] {
			\scriptsize\textcolor{blue}{\url{https://github.com/oliviercailloux/IDH/}}
		};
	}
	\titlepage
\end{frame}
\addtocounter{framenumber}{-1}

\begin{frame}
	\frametitle{Plan}
	\tableofcontents[hideallsubsections, sectionstyle=shaded/show]
\end{frame}

\AtBeginSection{
	\begin{frame}
		\frametitle{Plan}
		\tableofcontents[currentsection, hideallsubsections]
	\end{frame}
}

\section{Histoire}
\subsection{Avant l’IDH}
\begin{frame}
	\frametitle{La préhistoire de l’IDH}
  \begin{itemize}
    \item Marginalistes (\onslide<1->{date ?} \onslide<2->{\textasciitilde{} 187x}) : pas de comparaison d’utilité entre individus
    \item Révolution \og{}ordinaliste\fg{}, Pareto : comparaison uniquement par Pareto-dominance
    \item Mais souhait d’optimisation et de comparaison néanmoins !
    \item En pratique, usage du PIB ou variantes (\onslide<1->{date ?} \onslide<3->{\textasciitilde{} 193x})
  \end{itemize}
\end{frame}

\begin{frame}
	\frametitle{Critiques amenant à l’IDH : le PIB}
  \begin{block}{L’insuffisance du PIB}
    \begin{itemize}
      \item \emph{the welfare of a nation can scarcely be inferred from a measure of national income} {\small (Kuznets 1934, cité par \href{https://en.wikipedia.org/wiki/Simon_Kuznets\#National_income_accounts}{Wikipedia})} %Simon Kuznets, 1934. "National Income, 1929–1932". 73rd US Congress, 2d session, Senate document no. 124, page 7. https://fraser.stlouisfed.org/title/national-income-1929-1932-971
      \item \emph{You could be well off, without being well. You could be well, without being able to lead the life you wanted.} {\small (Sen 1987a)}% The Standard of Living, Sen (1987b: 1).
    \end{itemize}
  \end{block}
\end{frame}

\begin{frame}
	\frametitle{Critiques amenant à l’IDH : l’approche}
  \begin{block}{L’étroitesse de l’approche}
    \begin{itemize}
      \item \emph{ attempting to maximize economic growth without paying any direct attention to the transformation of greater opulence into better living conditions (…), in general, is a roundabout, undependable, and wasteful way of improving the living standards of the poor}
      \item \emph{with Pareto optimality as the only criterion of judgement, and self-seeking behaviour as the only basis of economic choice, the scope for saying something interesting in welfare economics became exceedingly small}
    \end{itemize}
    {\small (Sen 1989, 1987b, cités par Stanton)}
  \end{block}
\end{frame}

\begin{frame}
	\frametitle{Théories amenant à l’IDH}
  \begin{itemize}
    \item Philosophie de la justice ; Rawls (1971), distribution originelle de biens primaires vue comme des moyens
    \item Sen \& Nussbaum : capabilités, fonctionnements
  \end{itemize}
\end{frame}

\begin{frame}
	\frametitle{Au-delà du PIB}
	\begin{itemize}
		\item 1966, UNRISD (? \onslide<2->{{\tiny Institut de recherche des Nations unies pour le développement social}}), “level of living index” : nutrition, abri, santé ; éducation, loisirs, sécurité ; revenu…
		% \item McGranahan (1972), UNRISD, “Development Index” : 9 caractéristiques économiques et 9 sociales
		% \item 1975, United Nations Economic and Social Council, 7 indicateurs : alphabétisation, espérance de vie, énergie, part de l’industrie dans le PIB, part de l’industrie dans les exportations, emploi hors agriculture, nombre de téléphones
		\item 1976, International Labor Organization, “basic needs” : santé, éducation, services essentiels, consommation
		\item 1979, Physical Quality of Life Index (PQLI) : mortalité infantile, espérance de vie à un an, alphabétisation
		\item 198x : observation que l’espérance de vie est fortement corrélée à beaucoup d’indicateurs
		% \item 1987, 10 indicateurs
		\item 1991, 20 indicateurs analysés, concluent que trois suffisent
	\end{itemize}
  Foison de propositions et d’approches…
\end{frame}

\subsection{L’IDH et son évolution}
\begin{frame}
	\frametitle{L’IDH}
	\begin{itemize}
		\item PNUD (? \onslide<2->{{\tiny Programme des Nations Unies pour le Développement}}), 1965 : éliminer la pauvreté ; développement soutenable et humain
		\item Rapport annuel pour le PNUD, à partir de 1990
		\item \href{https://hdr.undp.org/content/human-development-report-1990}{Premier rapport} : première définition de l’Indice de Développement Humain (IDH)
		\item Par Mahbub ul Haq, avec consultants (dont ? \onslide<3->{Amartya Sen})
		\item Trois sous-indicateurs : espérance de vie, éducation, PIB par habitant (log), aggrégés en un indice
	\end{itemize}
\end{frame}

\begin{frame}
	\frametitle{L’évolution de l’IDH}
	\begin{itemize}
		\item 1991 : éducation comprend l’âge moyen de scolarisation ; changement transformation du PIB par habitant
		\item 1994 : bornes des sous-indicateurs fixées
		\item 1995 : éducation mesurée différemment ; changement d’une borne
		\item 1999 : changement transformation du PIB par habitant (ln avec bornes)
		\item 2010 : Indice de développement humain ajusté aux inégalités (IDHI), Indice de pauvreté multidimensionnelle (IPM), Indice d’inégalité de genre (IIG)
		\item 2014 : Indice de développement de genre (IDG)
		\item 2020 : IDH ajusté aux pressions exercées sur la planète (IDHP) %vérifié date avec rapports 2019 et 2020
    \item Intégrer \href{https://hdr.undp.org/system/files/documents/2018humandevelopmentstatisticalupdatefr.pdf}{figure 1, Évolution des indices composés de développement humain, Bureau du Rapport sur le développement humain}
	\end{itemize}
\end{frame}

\section{Définitions, …}

\section{Limites}

\begin{frame}[plain]
	\addtocounter{framenumber}{-1}
	\begin{center}
		\huge
		\textit{Meci pour votre attention !}
	\end{center}
\end{frame}

\end{document}

\appendix
\AtBeginSection{
}

\begin{frame}[allowframebreaks]
	\frametitle{\refname}
 	\bibliography{zotero}
\end{frame}

\clearpage\pdfbookmark{License}{License}
\begin{frame}[plain]
	\frametitle{License}
	This presentation, and the associated \LaTeX{} code, are published under the \href{https://opensource.org/licenses/MIT}{MIT license}. Feel free to reuse (parts of) the presentation, under condition that you cite the author.
	
	Credits are to be given to the authors.
\end{frame}
\addtocounter{framenumber}{-1}
\end{document}

\begin{frame}
	\frametitle{Title}
	\begin{itemize}
		\item Item
	\end{itemize}
\end{frame}

\begin{frame}
	\frametitle{Title}
	\begin{block}{Block}
%		\setlength\abovedisplayskip{1 ex}% reduce space above equations
		\begin{itemize}
			\item Item
		\end{itemize}
	\end{block}
	\begin{itemize}
		\item Item
	\end{itemize}
\end{frame}

